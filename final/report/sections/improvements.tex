\section{Future Improvements}

\subsection{Multi-resolution Active Shape Model}
To improve the efficiency and robustness of the detection, a
multi-resolution model can be used.

At the base of the pyramid it is the original image and
the level is the lowest (level 0). The image in the higher
level is formed by subsampling the former image then we obtain
a lower resolution version of the image with half number of
the pixels along each dimension. Subsequent levels are
obtained by further subsampling.

With this method, the initial fit of the model to approximate
will vary faster in position and orientation, and this fit will be
improved and refinated in each level of the pyramid.

\subsection{Width of Search Profile}
In the implemented algorithm, search profile is only
along a single line when searching forthe suggested
movements of landmarks. But this search technique could
be affected by noise in the image. To avoid this problem,
the search space could be expanded adding some width to it. 

\subsection{Landmarks Grouping}
As we have seen new allowed shapes should comply with
the constraint that the range of $b$ must be in the inverval
$-3\sqrt\lambda \leq b \leq 3\sqrt\lambda$, where $\lambda$ is the
eigenvalue result of the principal component analysis. But some of
the shape variation isnt decided by a single principal component, and
unexpected shapes could still apear even with the range of $b$ limited.
This problem can be avoided grouping ladmarks that tend to move together.

\subsection{Automatic model}

There are several factors that could improve the automatic model, some of them
are:
\begin{itemize}
  \item If the current classifier is used, the position of the upper incisives
could be slightly improved by taking into account the fact that the upper
central right and left teeth are always bigger than the right and left laterals,
and accordingly set an initial $x$ position for them
  \item Again, given our current classifier, the position of the lower
incisives, instead of making a division of 8 areas, a division of 6 areas could
improve the results.
  \item Lastly, instead of meddling with cumbersomes matters such as trying to
calculate the position of upper and lower incisives, given that there is one big
rectangle, it could be used another classifier, one for detecting upper
incisives, another for lower incisives. That could increase detection accuracy,
and boost shape placing.
\end{itemize}

